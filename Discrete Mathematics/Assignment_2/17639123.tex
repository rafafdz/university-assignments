% Plantilla para documentos LaTeX para enunciados
% Por Pedro Pablo Aste Kompen - ppaste@uc.cl
% Licencia Creative Commons BY-NC-SA 3.0
% http://creativecommons.org/licenses/by-nc-sa/3.0/

\documentclass[12pt]{article}

% Margen de 1 pulgada por lado
\usepackage{fullpage}
% Incluye gráficas
\usepackage{graphicx}
% Packages para matemáticas, por la American Mathematical Society
\usepackage{amssymb}
\usepackage{amsmath}
% Desactivar hyphenation
\usepackage[none]{hyphenat}
% Saltar entre párrafos - sin sangrías
\usepackage{parskip}
% Español y UTF-8
\usepackage[spanish]{babel}
\usepackage[utf8]{inputenc}
% Links en el documento
\usepackage{hyperref}
\usepackage{fancyhdr}
\usepackage{enumitem}
\setlength{\headheight}{15.2pt}
\setlength{\headsep}{5pt}
\pagestyle{fancy}
\usepackage{mathtools}
\DeclarePairedDelimiter{\ceil}{\lceil}{\rceil}

\newcommand{\N}{\mathbb{N}}
\newcommand{\Exp}[1]{\mathcal{E}_{#1}}
\newcommand{\List}[1]{\mathcal{L}_{#1}}
\newcommand{\EN}{\Exp{\N}}
\newcommand{\LN}{\List{\N}}

\newcommand{\comment}[1]{}
\newcommand{\lb}{\\~\\}
\newcommand{\eop}{_{\square}}
\newcommand{\hsig}{\hat{\sigma}}
\newcommand{\ra}{\rightarrow}
\newcommand{\lra}{\leftrightarrow}

% Cambiar por nombre completo + número de alumno
\newcommand{\alumno}{Rafael Fernández - 17639123}
\rhead{Tarea 2 - \alumno}

\begin{document}


\thispagestyle{empty}
% Membrete
% PUC-ING-DCC-IIC1103
\begin{minipage}{2.3cm}
\includegraphics[width=2cm]{img/logo.pdf}
\vspace{0.5cm} % Altura de la corona del logo, así el texto queda alineado verticalmente con el círculo del logo.
\end{minipage}
\begin{minipage}{\linewidth}
\textsc{\raggedright \footnotesize
Pontificia Universidad Católica de Chile \\
Departamento de Ciencia de la Computación \\
IIC1253 - Matemáticas Discretas \\}
\end{minipage}


% Titulo
\begin{center}
\vspace{0.5cm}
{\huge\bf Tarea 2}\\
\vspace{0.2cm}
\today\\
\vspace{0.2cm}
\footnotesize{2º semestre 2019 - Profesores G. Diéguez - F. Suárez}\\
\vspace{0.2cm}
\footnotesize{\alumno}
\rule{\textwidth}{0.05mm}
\end{center}



\section*{Respuestas}
% Estas numeracion es solo de ejemplo

\subsection*{Pregunta 1}

a)\\

Ya que sabemos que $\{\neg\textrm{,} \vee\}$ es funcionalmente completo, demostraremos con inducción estructural que para cada formula construida con $\{\neg\textrm{,}\vee\}$, existe otra fórmula equivalente con $\{\sim\textrm{,}\rightarrow\}$.\\

CB: $\varphi = p$, se cumple trivialmente. \\
HI: Si $\alpha$, $\beta$ se pueden contruir con $\{\neg\textrm{,} \vee\}$, entonces existen $\alpha'$ y $\beta$' escritas con $\{\sim\textrm{,}\rightarrow\}$ tal que $\alpha'\equiv\alpha$ y $\beta'\equiv\beta$.\\
TI: Paso inductivo
\begin{itemize}
  \item $\omega\equiv\neg\alpha\equiv\alpha'\equiv\alpha\rightarrow\sim\alpha\equiv\omega'$\\
  \item $\omega\equiv\alpha\vee\beta\equiv\alpha'\vee\beta'\equiv\neg\alpha\rightarrow\beta\equiv(\alpha\rightarrow\sim\alpha)\rightarrow\beta$
\end{itemize}\\

b)\\

Demostraremos que $\{+\textrm{,}\rightarrow\}$ no es FC por contradicción. Para esto se demostrará que para cada fórumla lógica $\varphi \in L(P)$ formada con $\{+\textrm{,}\rightarrow\}$ se cumple que $\varphi(p) \equiv p$ o $\varphi(p) \equiv  \top$

CB: $\varphi = p$, se cumple trivialmente. \\
HI: Si $\varphi_1$, $\varphi_2 \in L(P)$ escritos solo con $\{+\textrm{,}\rightarrow\}$, entonces $\varphi_1 \equiv \alpha$ y $\varphi_2 \equiv \alpha$, con $\alpha \in \{p \textrm{,}\top \}$\\
TI: Paso inductivo
\begin{itemize}
  \item $+\varphi_1 \equiv \top$ por definición\\
  \item Para el conector $\rightarrow$ lo demostraremos caso por caso.
\[
\begin{tabular}{cc|c}
$\varphi_1$ & $\varphi_2$ & $\varphi_1\rightarrow\varphi_2$ \\
\hline
$p$ & $p$ & $\top$ \\
$p$ & $\top$ & $\top$ \\
$\top$ & $p$ & $p$ \\
$\top$ & $\top$ & $\top$\\
\end{tabular}
\]
\end{itemize}

Como se puede ver, es imposible obtener la negación $\varphi(p)\equiv\neg p$, por lo que el conjunto $\{+\textrm{,}\rightarrow\}$ no puede ser funcionalmente completo.

\newpage
\subsection*{Pregunta 2}

a)\\

Demostraremos por resolución con $\Sigma = \{\exists x (A(x)) \vee \exists y (B(y)) \textrm{,}\forall x (A(x) \rightarrow B(x)) \textrm{,} \neg(\exists y (B(y)))\} \equiv \square$

\begin{enumerate}[label={(\arabic*)}]
\item $\forall y(\neg B(y))$
\item $\neg A(a) \vee B(a)$
\item $\neg B(a)$
\item $\neg A(a)$
\item $A(a) \vee B(a)$
\item $A(a)$
\item $\square$
\end{enumerate}


b)\\
\begin{enumerate}[label={(\arabic*)}]
\item $A(a)\wedge \neg C(b)$
\item $A(b)\wedge \neg C(c)$
\item $\neg A(b) \vee B(b)$
\item $\neg B(b) \vee C(b)$
\item $\neg B(b)$
\item $\neg A(b)$
\item $\square$
\end{enumerate}


% Fin del documento
\end{document}