% Plantilla para documentos LaTeX para enunciados
% Por Pedro Pablo Aste Kompen - ppaste@uc.cl
% Licencia Creative Commons BY-NC-SA 3.0
% http://creativecommons.org/licenses/by-nc-sa/3.0/

\documentclass[12pt]{article}

% Margen de 1 pulgada por lado
\usepackage{fullpage}
% Incluye gráficas
\usepackage{graphicx}
% Packages para matemáticas, por la American Mathematical Society
\usepackage{amssymb}
\usepackage{amsmath}
% Desactivar hyphenation
\usepackage[none]{hyphenat}
% Saltar entre párrafos - sin sangrías
\usepackage{parskip}
% Español y UTF-8
\usepackage[spanish]{babel}
\usepackage[utf8]{inputenc}
% Links en el documento
\usepackage{hyperref}
\usepackage{fancyhdr}
\setlength{\headheight}{15.2pt}
\setlength{\headsep}{5pt}
\pagestyle{fancy}
\usepackage{mathtools}
\DeclarePairedDelimiter{\ceil}{\lceil}{\rceil}

\newcommand{\N}{\mathbb{N}}
\newcommand{\Exp}[1]{\mathcal{E}_{#1}}
\newcommand{\List}[1]{\mathcal{L}_{#1}}
\newcommand{\EN}{\Exp{\N}}
\newcommand{\LN}{\List{\N}}

\newcommand{\comment}[1]{}
\newcommand{\lb}{\\~\\}
\newcommand{\eop}{_{\square}}
\newcommand{\hsig}{\hat{\sigma}}
\newcommand{\ra}{\rightarrow}
\newcommand{\lra}{\leftrightarrow}

% Cambiar por nombre completo + número de alumno
\newcommand{\alumno}{Rafael Fernández - 17639123}
\rhead{Tarea 1 - \alumno}

\begin{document}


\thispagestyle{empty}
% Membrete
% PUC-ING-DCC-IIC1103
\begin{minipage}{2.3cm}
\includegraphics[width=2cm]{img/logo.pdf}
\vspace{0.5cm} % Altura de la corona del logo, así el texto queda alineado verticalmente con el círculo del logo.
\end{minipage}
\begin{minipage}{\linewidth}
\textsc{\raggedright \footnotesize
Pontificia Universidad Católica de Chile \\
Departamento de Ciencia de la Computación \\
IIC1253 - Matemáticas Discretas \\}
\end{minipage}


% Titulo
\begin{center}
\vspace{0.5cm}
{\huge\bf Tarea 1}\\
\vspace{0.2cm}
\today\\
\vspace{0.2cm}
\footnotesize{2º semestre 2019 - Profesores G. Diéguez - F. Suárez}\\
\vspace{0.2cm}
\footnotesize{\alumno}
\rule{\textwidth}{0.05mm}
\end{center}



\section*{Respuestas}
% Estas numeracion es solo de ejemplo

\subsection*{Pregunta 1}
\textbf{Por Demostrar:}
\\
\[\sum_{i=0}^{n} i(-1)^i = (-1)^n \ceil[\Big]{\frac{n}{2}},  \forall n \in \mathbb{N}\] \\
\textbf{Caso Base}:
\\
\[0 + (-1)^1 * 1 = 1 = (-1)^1 * \ceil[\Big]{\frac{1}{2}} \]
\\
\textbf{Hipotesis Inductiva}: Suponemos que la siguiente ecuación es cierta:
\[\sum_{i=0}^{n} i(-1)^i = (-1)^n \ceil[\Big]{\frac{n}{2}},  \forall n \in \mathbb{N}\] 

\newpage
\textbf{Tesis Inductiva}:

\begin{align}
\sum_{i=0}^{n + 1} i(-1)^i&=(-1)^{n+1} \ceil[\Big]{\frac{n+1}{2}}\\
\sum_{i=0}^{n} i(-1)^i + (-1)^{n+1}(n+1)&=\\
(-1)^{n} \ceil[\Big]{\frac{n}{2}}-(-1)^{n}(n+1)&=\\
(-1)^{n}(\ceil[\Big]{\frac{n}{2}}-(n+1))&=\\
(-1)^{n+1}((n+1)-\ceil[\Big]{\frac{n}{2}})&=
\end{align}

\textbf{Caso $n$ es par}:\\
\text{Sea $n = 2k$}\\
\begin{align}
(-1)^{2k+1}((2k+1)-\ceil[\Big]{\frac{2k}{2}})&=\\
(-1)^{2k+1}((2k+1)-k)&=\\
(-1)^{2k+1}(k+1)&=\\
(-1)^{2k+1}\ceil[\Big]{\frac{2k +1}{2}}&=(-1)^{n+1}\ceil[\Big]{\frac{n +1}{2}}\\
\end{align}
\textbf{Caso $n$ es impar}:\\
\text{Sea $n = 2k + 1$}\\
\begin{align}
(-1)^{2k+2}((2k+2)-\ceil[\Big]{\frac{2k+1}{2}})&=\\
(-1)^{2k+2}((2k+2)-(k + 1))&=\\
(-1)^{2k+2}(k+1)&=\\
(-1)^{2k+2}\ceil[\Big]{\frac{2k +1}{2}}&=(-1)^{n+1}\ceil[\Big]{\frac{n +1}{2}}\\
\end{align}


\newpage
\subsection*{Pregunta 2}

Sea $S$ el conjunto de todos los Strings definidos por:\\
\begin{math}
0 \in S \\
1 \in S \\
x \in S, y \in S \Rightarrow x y \in S
\end{math}

Nota: De esta definición se desprende que $L \subset S$ \\

Sea la función $Unos: S \rightarrow \mathbb{N}$ definida por: \\
\begin{math}
Unos(0) = 0 \\
Unos(1) = 1 \\
Unos(xy) = Unos(x) + Unos(y)\\
\end{math}

\textbf{Por Demostrar:}\\
\[\exists n \in \mathbb{N} | Unos(x) = 2n + 1 \Leftrightarrow x \in L, x \in S\] \\

\textbf{Sentido $\Rightarrow$:}\\

\textbf{Por Demostrar:}\\
\[\exists n \in \mathbb{N} | Unos(x) = 2n + 1 \Rightarrow x \in L, x \in S\] \\


\textbf{Hipotesis Inductiva}: Suponemos que la siguiente implicancia es cierta:
\[\exists n \in \mathbb{N} | Unos(x) = 2n + 1 \Rightarrow x \in L, x \in S\] \\
\textbf{Caso Base}:\\
\begin{math}
s = 1, Unos(s) = 1 = 2*0 + 1, s \in L \textrm{ por definición}\\ \\
\end{math}
\textbf{Tesis Inductiva}:\\
Sea $s \in S$

Caso en que $s$ es de la forma $0x$:\\
\begin{math}
Unos(0x) = 2n + 1 = Unos(x) \textrm{, por lo tanto}, \\
x \in L \textrm{ por HI y } 0x \in L \textrm{ por definición}\\
\end{math} \\

Caso en que $s$ es de la forma $x0$:\\
\begin{math}
Unos(x0) = 2n + 1 = Unos(x) \textrm{, por lo tanto}, \\
x \in L \textrm{ por HI y } x0 \in L \textrm{ por definición}\\
\end{math} \\

Caso en que $s$ es de la forma $x1y$:\\
\begin{math}
Unos(x1y) = 2n + 1 \Rightarrow Unos(xy) \textrm{ es par} \Rightarrow Unos(x) + Unos(y)\textrm{ es par}, \\
\textrm{Se puede tener un } x | Unos(x) \textrm{ es impar y un } y | Unos(y) \textrm{ es impar. Luego,} \\
x,y \in L \textrm{ por HI y } x1y \in L \textrm{ por definición}. 
\end{math} \\

\textbf{Sentido $\Leftarrow$:}\\

\textbf{Por Demostrar:}\\
\[x \in L \Rightarrow \exists n \in \mathbb{N} | Unos(x) = 2n + 1\] \\


\textbf{Hipotesis Inductiva}: Suponemos que la siguiente implicancia es cierta:
\[x \in L \Rightarrow \exists n \in \mathbb{N} | Unos(x) = 2n + 1\] \\
\textbf{Caso Base}:\\
\begin{math}
s = 1, s \in L,  Unos(s) = 1 = 2*0 + 1\\ \\
\end{math}
\textbf{Tesis Inductiva}:\\
1. $x \in L \Rightarrow 0x \in L$: \\
\begin{math}
Unos(x) = 2n + 1\\
Unos(0x) = 0 + Unos(x) = 2n + 1\\
\end{math}
\\
2. $x \in L \Rightarrow x0 \in L$: \\
\begin{math}
Unos(x) = 2n + 1\\
Unos(x0) = Unos(x) + 0 = 2n + 1\\
\end{math}
\newpage
3. $x,y \in L \Rightarrow x1y \in L$: \\
\begin{math}
Unos(x1y) = Unos(x) + 1 + Unos(y)\\
Unos(x1y) = 2p + 1 + 1 + 2q + 1; p,q \in \mathbb{N}, \textrm{ por HI.}\\
Unos(x1y) = 2p + 2q + 2 +1 \\
Unos(x1y) = 2p + 2q + 2 +1 \\
Unos(x1y) = 2(p + q + 1) + 1 \\
Unos(x1y) = 2r + 1, r \in \mathbb{N} \\ 
\end{math}

% Fin del documento
\end{document}