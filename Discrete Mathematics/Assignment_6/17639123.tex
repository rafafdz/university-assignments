% Plantilla para documentos LaTeX para enunciados
% Por Pedro Pablo Aste Kompen - ppaste@uc.cl
% Licencia Creative Commons BY-NC-SA 3.0
% http://creativecommons.org/licenses/by-nc-sa/3.0/

\documentclass[12pt]{article}

% Margen de 1 pulgada por lado
\usepackage{fullpage}
% Incluye gráficas
\usepackage{graphicx}
% Packages para matemáticas, por la American Mathematical Society
\usepackage{amssymb}
\usepackage{amsmath}
% Desactivar hyphenation
\usepackage[none]{hyphenat}
% Saltar entre párrafos - sin sangrías
\usepackage{parskip}
% Español y UTF-8
\usepackage[spanish]{babel}
\usepackage[utf8]{inputenc}
% Links en el documento
\usepackage{hyperref}
\usepackage{fancyhdr}
\setlength{\headheight}{15.2pt}
\setlength{\headsep}{5pt}
\pagestyle{fancy}

\newcommand{\N}{\mathbb{N}}
\newcommand{\Exp}[1]{\mathcal{E}_{#1}}
\newcommand{\List}[1]{\mathcal{L}_{#1}}
\newcommand{\EN}{\Exp{\N}}
\newcommand{\LN}{\List{\N}}

\newcommand{\comment}[1]{}
\newcommand{\lb}{\\~\\}
\newcommand{\eop}{_{\square}}
\newcommand{\hsig}{\hat{\sigma}}
\newcommand{\ra}{\rightarrow}
\newcommand{\lra}{\leftrightarrow}

% Cambiar por nombre completo + número de alumno
\newcommand{\alumno}{Rafael Fernández - 17639123}
\rhead{Tarea 6 - \alumno}

\begin{document}
\thispagestyle{empty}
% Membrete
% PUC-ING-DCC-IIC1103
\begin{minipage}{2.3cm}
\includegraphics[width=2cm]{img/logo.pdf}
\vspace{0.5cm} % Altura de la corona del logo, así el texto queda alineado verticalmente con el círculo del logo.
\end{minipage}
\begin{minipage}{\linewidth}
\textsc{\raggedright \footnotesize
Pontificia Universidad Católica de Chile \\
Departamento de Ciencia de la Computación \\
IIC1253 - Matemáticas Discretas \\}
\end{minipage}


% Titulo
\begin{center}
\vspace{0.5cm}
{\huge\bf Tarea 6}\\
\vspace{0.2cm}
\today\\
\vspace{0.2cm}
\footnotesize{2º semestre 2019 - Profesores G. Diéguez - F. Suárez}\\
\vspace{0.2cm}
\footnotesize{\alumno}
\rule{\textwidth}{0.05mm}
\end{center}



\section*{Respuestas}
% Estas numeracion es solo de ejemplo

\subsection*{Pregunta 1}

Sean $G$ y $H$ grafos simples con \textit{caminos Hamiltonianos}. \\
Sea $C_G = (g_1, g_2, ..., g_n)$ con $g_i \in V(G), i \in \{1..n\}$ un \textit{Camino Hamiltoniano} en $G$. \\
Sea $C_H = (h_1, h_2, ..., h_m)$ con $h_i \in V(H), i \in \{1..m\}$ un \textit{Camino Hamiltoniano} en $H$. \\

Sea $I = G \times V$. Por definición de producto cartesiano entre grafos, tenemos los siguientes \\
caminos en $I$:
\begin{align*}
C_1 &= ((g_1, h_1), (g_1, h_2), ... , (g_1, h_m)) \\
C_2 &= ((g_2, h_m), (g_2, h_{m-1}), ... , (g_2, h_1)) \\
C_3 &= ((g_3, h_1), (g_3, h_2), ... , (g_3, h_m)) \\
\vdots \\
C_n &= \begin{cases} 
      ((g_n, h_1), (g_n, h_2), ... , (g_n, h_m)) & n \textrm{ impar} \\
      ((g_n, h_m), (g_n, h_{m-1}), ... , (g_n, h_1)) & n \textrm{ par}  
\end{cases}
\end{align*}

\newpage

Luego, entre cada camino $C_i, C_{i+1}, \forall i \in \{1,...,n-1\}$, por la definición del producto cruz, existen aristas de la forma: \\
\[
a_{i,i+1}
\begin{cases} 
      ((g_i, h_m), (g_{i+1}, h_m)) & i \textrm{ impar} \\
      ((g_i, h_1), (g_{i+1}, h_1)) & i \textrm{ par}  
\end{cases}
\]
Por lo tanto, tenemos el camino $C = C_1, a_{1,2}, C_2, a_{2,3}, ..., C_{n-1}, a_{n-1, n}, C_n$, el cual es \textit{Hamiltoniano} ya que contiene a todos los vértices de $I$.

$$\blacksquare$$


\newpage
\subsection*{Pregunta 2}

\subsubsection*{Pregunta 2.a}
Por demostrar: \\
$G$ es árbol $\Leftrightarrow$ tiene exactamente un ciclo al agregar una arista cualquiera 

$(\Rightarrow)$ \\
Sea $v_1, v_2 \in V(G)$. Ya que $G$ es aŕbol, existe un camino único $v_2, ..., v_2$ que conecta ambos vértices. Luego, al agregar la arista $(v_2, v_1)$ se habrá formado un ciclo. Ya que el camino que unía a los vértices era único, el ciclo es único. $\blacksquare$. \\

$(\Leftarrow)$
Sea G un grafo que al agregar una arista cualquiera se forma exactamente un ciclo. Ya que al agregar una arista $(v_1, v_2)$, $v_1, v_2 \in V(G)$, se forma un ciclo, entonces debe existir un camino $v_2, ..., v_1$ entre ambos vertices. Ya que el ciclo es único, el camino debe ser único. Como elegimos $v_1, v_2$ genéricos, esto se cumple para todos los vértices, por lo que hay un camino único entre cada par de vértices $\Rightarrow G$ es un árbol. $\blacksquare$. \\

\subsubsection*{Pregunta 2.b}
% Respuesta pregunta
Sea $T$ un bosque con $k$ árboles (componentes conexas). \\
Luego, $|V(T)| = n = \displaystyle\sum_{i=1}^{k}{|V(k_i)|}$, donde $k_i$ es el i-esimo árbol. \\
Por la definición alternativa vista en clases, sabemos que cada árbol tiene $|V(k_i)| - 1$ aristas.
Ya que no hay caminos entre árboles, no hay aristas entre estos y la cantidad total de aristas es la siguiente: \\
\begin{align*}
|E(T)| &= \sum_{i=1}^k(|V(k_i)| - 1) \\
&= \sum_{i=1}^{k}{|V(k_i)|} - \sum_{i=1}^{k} 1 \\ 
&= n - k \\
\blacksquare
\end{align*}

\newpage
\subsubsection*{Pregunta 2.c}
Sea $G = (V, E)$ un grafo cualquiera. Sea $v \in V(G)$ un vértice de grado $k > 1$. \\
Sabemos por teorema visto en clases que los árboles son estructuras recursivas ($T - v$ también es árbol). \\
Luego, para cada uno de los vértices adyacetes a $v$ se tiene un árbol. 
Sabemos que un árbol no vacío tiene al menos un nodo hoja.

En el caso de que $v$ sea nodo raíz, tiene $k$ hijos y por lo tanto $k$ sub-árboles, lo que implica que tiene al menos $k$ nodos hojas.\\
En otro caso, $v$ tiene $k - 1$ hijos , lo que significa al menos $k - 1$ hojas
Ya que cada árbol tiene por lo menos 1 nodo hoja, tenemos por lo menos $k - 1$ hojas. Luego hay 2 casos:
\begin{itemize}
  \item Grado del nodo padre de $v$ es 1 $\Rightarrow$ es nodo hoja $\Rightarrow$ al menos $k$ hojas en total.
  \item Grado nodo padre es mayor a 1 $\Rightarrow$ existe al menos otro sub-árbol hermano de $v$ $\Rightarrow$ al menos $k$ hojas en total.
\end{itemize}
$$\blacksquare$$
% Fin del documento
\end{document}
