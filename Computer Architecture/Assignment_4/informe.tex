\documentclass{article}
\usepackage[utf8]{inputenc}
\usepackage{float}
\restylefloat{table}

\title{%
  \vspace{-40mm}
  Informe Tarea 4  \\
  \large Arquitectura de Computadores}
\author{Rafael Fernández}

\begin{document}
\date{}

\maketitle

\section{Desarrollo}
Para el presente informe se desarrolló un \textit{software} capaz de simular los accesos a memoria de una serie de programas entregados en formato \textit{.json}. El software simula los estados de la memoria física, la caché, la TLB y la tabla de páginas de cada programa en cada acceso a memoria. Los parámetros de cada uno de los componentes mencionados anteriormente, correspondientes a políticas de reemplazo, tamaños en \textit{bytes}, entre otros, se entregan como \textit{inputs} al programa.
En base a dicha información, el programa genera una serie de estadísticas por programa. Estas serán estudiadas a continuación para determinar de manera empírica cuales son los parámetros que generan un mejor rendimiento.

Se empezará por estudiar el comportamiento del conjunto de programas del archivo \textit{test1.json}. Las estadisticas resultantes de la simulación con los parámetros por defecto del archivo fueron los siguientes:
\begin{table}[H]
\begin{tabular}{|l|l|l|l|l|l|l|l|l|l|}
\hline
\multicolumn{1}{|c|}{\textbf{Programa}} & \multicolumn{1}{c|}{\textbf{HRT}} & \multicolumn{1}{c|}{\textbf{HRC}} & \multicolumn{1}{c|}{\textbf{PF}} & \multicolumn{1}{c|}{\textbf{SO}} & \multicolumn{1}{c|}{\textbf{SI}} & \multicolumn{1}{c|}{\textbf{ST}} & \multicolumn{1}{c|}{\textbf{TT}} & \multicolumn{1}{c|}{\textbf{CT}} & \multicolumn{1}{c|}{\textbf{MT}} \\ \hline
Paint                                   & 0.73                              & 0.64                              & 1                                & 0                                & 0                                & 0                                & 2230                             & 1352                             & 2150                             \\ \hline
Chrome                                  & 0.8                               & 0.6                               & 1                                & 0                                & 0                                & 0                                & 2020                             & 1232                             & 2150                             \\ \hline
Telegram                                & 0.8                               & 0.4                               & 1                                & 0                                & 0                                & 0                                & 1010                             & 624                              & 1650                             \\ \hline
Sublime                                 & 0.86                              & 0.43                              & 1                                & 0                                & 0                                & 0                                & 1410                             & 872                              & 2150                             \\ \hline
\end{tabular}
\end{table}

Para empezar,  cambiaremos  la política de reemplazo de la TLB de LFU a LRU.

\begin{table}[H]
\begin{tabular}{|l|l|l|l|l|l|l|l|l|l|}
\hline
\multicolumn{1}{|c|}{\textbf{Programa}} & \multicolumn{1}{c|}{\textbf{HRT}} & \multicolumn{1}{c|}{\textbf{HRC}} & \multicolumn{1}{c|}{\textbf{PF}} & \multicolumn{1}{c|}{\textbf{SO}} & \multicolumn{1}{c|}{\textbf{SI}} & \multicolumn{1}{c|}{\textbf{ST}} & \multicolumn{1}{c|}{\textbf{TT}} & \multicolumn{1}{c|}{\textbf{CT}} & \multicolumn{1}{c|}{\textbf{MT}} \\ \hline
Paint                                   & 0.73                              & 0.64                              & 1                                & 0                                & 0                                & 0                                & 2230                             & 1352                             & 2150                             \\ \hline
Chrome                                  & 0.8                               & 0.6                               & 1                                & 0                                & 0                                & 0                                & 2020                             & 1232                             & 2150                             \\ \hline
Telegram                                & 0.8                               & 0.4                               & 1                                & 0                                & 0                                & 0                                & 1010                             & 624                              & 1650                             \\ \hline
Sublime                                 & 0.86                              & 0.43                              & 1                                & 0                                & 0                                & 0                                & 1410                             & 872                              & 2150                             \\ \hline
\end{tabular}
\end{table}


Como podemos ver, no hubo cambios respecto a la versión original. Esto puede explicarse por el hecho de que los accesos a memoria virtual son direcciones que corresponden a la página cero de la memoria virtual (Ver hit rate alto de la TLB), por lo que no es necesario cambiar reemplazar páginas en la TLB ya que esta nunca se llena, provocando que los cambios en la política de reemplazo no afecten el rendimiento.

Si ahora cambiamos el valor de la función de correspondencia de la caché de \textit{fully associative} a \textit{2-way-associative} obtenemos:

\begin{table}[H]
\begin{tabular}{|l|l|l|l|l|l|l|l|l|l|}
\hline
\multicolumn{1}{|c|}{\textbf{Programa}} & \multicolumn{1}{c|}{\textbf{HRT}} & \multicolumn{1}{c|}{\textbf{HRC}} & \multicolumn{1}{c|}{\textbf{PF}} & \multicolumn{1}{c|}{\textbf{SO}} & \multicolumn{1}{c|}{\textbf{SI}} & \multicolumn{1}{c|}{\textbf{ST}} & \multicolumn{1}{c|}{\textbf{TT}} & \multicolumn{1}{c|}{\textbf{CT}} & \multicolumn{1}{c|}{\textbf{MT}} \\ \hline
Paint                                   & 0.73                              & 0.64                              & 1                                & 0                                & 0                                & 0                                & 2230                             & 1352                             & 2150                             \\ \hline
Chrome                                  & 0.8                               & 0.7                               & 1                                & 0                                & 0                                & 0                                & 2020                             & 1224                             & 1650                             \\ \hline
Telegram                                & 0.8                               & 0.4                               & 1                                & 0                                & 0                                & 0                                & 1010                             & 624                              & 1650                             \\ \hline
Sublime                                 & 0.86                              & 0.43                              & 1                                & 0                                & 0                                & 0                                & 1410                             & 872                              & 2150                             \\ \hline
\end{tabular}
\end{table}

Como se puede ver, el Hit Rate de la caché al usar Chrome mejoró de 0.6 a 0.7, permitiendo así que el tiempo total de memoria disminuyera en 500 nanosegundos, significando una mejora de un 23\% menos de tiempo.\\


Ya que en este caso tenemos una caché grande en comparación a los accesos totales de memoria (128 bytes vs 34 accesos), se procederá a reducir el tamaño de la caché a 8 bytes para poder apreciar como afectan realmente las políticas de reemplazo y funciones de correspondencia en la memoría caché. Los resultados luego del cambio fueron los siguientes:

\begin{table}[H]
\begin{tabular}{|l|l|l|l|l|l|l|l|l|l|}
\hline
\multicolumn{1}{|c|}{\textbf{Programa}} & \multicolumn{1}{c|}{\textbf{HRT}} & \multicolumn{1}{c|}{\textbf{HRC}} & \multicolumn{1}{c|}{\textbf{PF}} & \multicolumn{1}{c|}{\textbf{SO}} & \multicolumn{1}{c|}{\textbf{SI}} & \multicolumn{1}{c|}{\textbf{ST}} & \multicolumn{1}{c|}{\textbf{TT}} & \multicolumn{1}{c|}{\textbf{CT}} & \multicolumn{1}{c|}{\textbf{MT}} \\ \hline
Paint                                   & 0.73                              & 0.27                              & 1                                & 0                                & 0                                & 0                                & 2230                             & 1384                             & 4150                             \\ \hline
Chrome                                  & 0.8                               & 0.5                               & 1                                & 0                                & 0                                & 0                                & 2020                             & 1240                             & 2650                             \\ \hline
Telegram                                & 0.8                               & 0.4                               & 1                                & 0                                & 0                                & 0                                & 1010                             & 624                              & 1650                             \\ \hline
Sublime                                 & 0.86                              & 0.43                              & 1                                & 0                                & 0                                & 0                                & 1410                             & 872                              & 2150                             \\ \hline
\end{tabular}
\end{table}

Ahora que el espacio en la caché es escaso, los reemplazos se empiezan a hacer más comunes y el tiempo de una entrada en la caché es menor. Esto se ve reflejado en las bajas de los hit rates de caché, donde destaca la asociada al programa Paint, quien disminuyó su hit rate de 64\% a 27\%.\\

A continuación se probará cambiando la política de reemplazo de la caché de LRU a LFU. Se obtuvieron los siguientes resultados.


\begin{table}[H]
\begin{tabular}{|l|l|l|l|l|l|l|l|l|l|}
\hline
\multicolumn{1}{|c|}{\textbf{Programa}} & \multicolumn{1}{c|}{\textbf{HRT}} & \multicolumn{1}{c|}{\textbf{HRC}} & \multicolumn{1}{c|}{\textbf{PF}} & \multicolumn{1}{c|}{\textbf{SO}} & \multicolumn{1}{c|}{\textbf{SI}} & \multicolumn{1}{c|}{\textbf{ST}} & \multicolumn{1}{c|}{\textbf{TT}} & \multicolumn{1}{c|}{\textbf{CT}} & \multicolumn{1}{c|}{\textbf{MT}} \\ \hline
Paint                                   & 0.73                              & 0.27                              & 1                                & 0                                & 0                                & 0                                & 2230                             & 1384                             & 4150                             \\ \hline
Chrome                                  & 0.8                               & 0.5                               & 1                                & 0                                & 0                                & 0                                & 2020                             & 1240                             & 2650                             \\ \hline
Telegram                                & 0.8                               & 0.4                               & 1                                & 0                                & 0                                & 0                                & 1010                             & 624                              & 1650                             \\ \hline
Sublime                                 & 0.86                              & 0.29                              & 1                                & 0                                & 0                                & 0                                & 1410                             & 880                              & 2650                             \\ \hline
\end{tabular}
\end{table}

Se observa que disminuyó el hit rate de los accesos a la caché del programa Sublime, mientras los otros programas se mantuvieron iguales. Esto se puede asociar a que los accesos a memoria del programa Sublime no son tan repetitivos (como se podría dar en el caso de un ciclo), si no más bien que hace referencia a direcciones recientemente usadas, como podría serlo un largo procedimiento donde se van definiendo variables auxiliares para cada etapa. Se concluye que conviene LRU para este tipo de accesos.

Habiendo hecho un breve análisis de cómo afectan los parámetros a las estadísticas, ahora se analizará como afectaría el manejo de writes a las estadísticas.

Ahora nos enfocaremos en el archivo \textit{test3.json}, cuyos resultados de la simulación son los siguientes:

\begin{table}[H]
\begin{tabular}{|l|l|l|l|l|l|l|l|l|l|}
\hline
\multicolumn{1}{|c|}{\textbf{Programa}} & \multicolumn{1}{c|}{\textbf{HRT}} & \multicolumn{1}{c|}{\textbf{HRC}} & \multicolumn{1}{c|}{\textbf{PF}} & \multicolumn{1}{c|}{\textbf{SO}} & \multicolumn{1}{c|}{\textbf{SI}} & \multicolumn{1}{c|}{\textbf{ST}} & \multicolumn{1}{c|}{\textbf{TT}} & \multicolumn{1}{c|}{\textbf{CT}} & \multicolumn{1}{c|}{\textbf{MT}} \\ \hline
Solitare                                & 0.0                               & 0.0                               & 1                                & 0                                & 0                                & 0                                & 48                               & 192                              & 1200                             \\ \hline
Minesweeper                             & 0.0                               & 0.0                               & 1                                & 0                                & 0                                & 0                                & 96                               & 384                              & 1900                             \\ \hline
Full Tilt! Pinball                      & 0.6                               & 0.2                               & 2                                & 0                                & 0                                & 0                                & 192                              & 896                              & 3800                             \\ \hline
Solitare Spider                         & 0.33                              & 0.0                               & 2                                & 0                                & 0                                & 0                                & 128                              & 576                              & 3100                             \\ \hline
\end{tabular}
\end{table}

Los resultados mostrados anteriormente son con el manejo de writes por defecto de la simulación: \textit{Write-Back}.\\

A continuación se modificará este manejo por uno tipo \textit{Write-Trough} y comparán los resultados.
\begin{table}[H]
\begin{tabular}{|l|l|l|l|l|l|l|l|l|l|}
\hline
\multicolumn{1}{|c|}{\textbf{Programa}} & \multicolumn{1}{c|}{\textbf{HRT}} & \multicolumn{1}{c|}{\textbf{HRC}} & \multicolumn{1}{c|}{\textbf{PF}} & \multicolumn{1}{c|}{\textbf{SO}} & \multicolumn{1}{c|}{\textbf{SI}} & \multicolumn{1}{c|}{\textbf{ST}} & \multicolumn{1}{c|}{\textbf{TT}} & \multicolumn{1}{c|}{\textbf{CT}} & \multicolumn{1}{c|}{\textbf{MT}} \\ \hline
Solitare                                & 0.0                               & 0.0                               & 1                                & 0                                & 0                                & 0                                & 48                               & 192                              & 1200                             \\ \hline
Minesweeper                             & 0.0                               & 0.0                               & 1                                & 0                                & 0                                & 0                                & 96                               & 384                              & 1900                             \\ \hline
Full Tilt! Pinball                      & 0.6                               & 0.2                               & 2                                & 0                                & 0                                & 0                                & 192                              & 896                              & 4300                             \\ \hline
Solitare Spider                         & 0.33                              & 0.0                               & 2                                & 0                                & 0                                & 0                                & 128                              & 576                              & 3100                             \\ \hline
\end{tabular}
\end{table}

Si bien los resultados en esta prueba no son impactantes debido al bajo hit rate de la caché, todavía se pueden ver los efectos esperados: aumemento en el tiempo total de memoria, como se puede ver en el caso del tercer programa, el cual es el único con un caché hit rate mayor a cero.
Esto se puede explicar ya que, con \textit{Write-Through}, calquier escritura en la caché significa una escritura en Memoria física, mientras que con \textit{Write-Back} sólo se escribe en memoria al momento de reemplazar una dirección en la caché.
Se espera que con altos hit-rate de la caché el aumento de tiempo total de memoria sea mucho mayor.

\section{Bibliografía}

Para el presente informe se utilizaron, principalmente, los apuntes del curso IIC2343 'Arquitectura de Computadores' creados por Alejandro Echeverría y Hans-Albert Löbel. Capítulos 'Memoria Caché' y 'Multiprogramación', accesibles mediante:

\url{https://github.com/IIC2343/syllabus/tree/master/Apuntes}

También se obtuvo información del foro de ayuda de dicho curso:

\url{https://github.com/IIC2343/syllabus/issues}


\end{document}

